\chapter{Task 42: Public transport in UK}

\resp{Edoardo D'Amore}

\section{Dataset}

The objective of this task is to divide by city the dataset provided by  \cite{Gallotti2015}.

The tables considered are:
\begin{itemize}
    \item \texttt{nodes}: contains nodes corresponding to the stops in public transport.
    \item \texttt{edges}: contains the connections between the nodes.
    \item \texttt{stops}: contains geographical information about the nodes.
    \item \texttt{cities}: contains all cities in the United Kingdom with a population of at least 50 thousands people \cite{pop2026}.
\end{itemize}

\section{Data processing}

The first step has been the sanitization of the tables, which consisted mainly in reformatting specific columns in order to correctly cast them to numerical values.

In order to collect all nodes related to a city, the following pipeline has been followed:
\begin{enumerate}
    \item join the \texttt{nodes} and the \texttt{stops} tables on the \texttt{ATCOCode} attribute. If one the \texttt{cities} is present in either the \texttt{NatGazLocality}, the \texttt{ParentLocality} or the \texttt{GrandParentLocality} attributes, then it is assigned to the node.
    \item nodes that are within a 0.5Km radius from eachother are assigned to the same group. Groups are divided based on how many different cities their nodes are assigned to:
    \begin{itemize}
        \item 0: the group is discarded;
        \item 1: the group is classified as ``safe'';
        \item 2 or more: the group is classified as ``unsafe''.
    \end{itemize}
    \item nodes in the ``safe'' groups are all assigned to the same city.
    \item nodes in the ``unsafe'' groups are assigned to the same city of the node they are closest to.
\end{enumerate}

To obtain the edges it is sufficient to filter the \texttt{edges} table by considering only the links between two nodes previously retrieved.

For each city, two files are produced: \texttt{nodes.csv} and \texttt{edges.csv}.
Note that, for some cities, it was not possible to correctly retrieve any node by following this procedure.

\section{Basic analysis}

In this section we will consider the networks of three cities in the United Kingdom: London, Birmingham and Liverpool.

\paragraph{Degree distribution}
From figure \ref{fig:degree_dist} it is possible to observe that the three networks show similar degree distribution.
%($P(k) \sim k^{-\gamma}$), respectively with $\gamma_{\text{London}} = ...$, $\gamma_{\text{Birmingham}} = ...$ and $\gamma_{\text{Liverpool}} = ...$.

\begin{figure}[h!]
    \centering
    \includegraphics[width=0.7\textwidth]{images/task_42/degree_distributions.png}
    \caption{Degree distributions.}
    \label{fig:degree_dist}
\end{figure}

\paragraph{Assortativity coefficients}
Even though these are directed networks, most of the links are reciprocated, resulting in similar values for the assortativity coefficients (table \ref{tab:assortativity}).

\begin{table}[h!]
    \centering
    \begin{tabular}{|c|c|c|c|c|}
        \hline
        City & $r^{(\mathrm{out}, \mathrm{out})}$ & $r^{(\mathrm{in}, \mathrm{out})}$ & $r^{(\mathrm{out}, \mathrm{in})}$ & $r^{(\mathrm{in}, \mathrm{in})}$ \\
        \hline
        London & 0.28 & 0.27 & 0.28 & 0.27 \\
        Birmingham & 0.28 & 0.30 & 0.29 & 0.29 \\
        Liverpool & 0.26 & 0.28 & 0.26 & 0.25 \\
        \hline
    \end{tabular}
    \caption{Assortativity coefficients.}
    \label{tab:assortativity}
\end{table}

\paragraph{Centrality measures}
Figures \ref{fig:closeness} and \ref{fig:betweenness} show the closeness and betweenness centrality measures respectively, for each node in the three networks.

\begin{figure}[h!]
    \centering
    \includegraphics[width = \textwidth]{images/task_42/closeness.png}
    \caption{Closeness centrality.}
    \label{fig:closeness}
\end{figure}

\begin{figure}[h!]
    \centering
    \includegraphics[width=\textwidth]{images/task_42/betweenness.png}
    \caption{Betweenness centrality.}
    \label{fig:betweenness}
\end{figure}

\paragraph{Nearest-neighbors' average degree}
In figure \ref{fig:nn_ad_dist} we can observe how the results are quite analougous for the three cities, with London having on average higher values.

\begin{figure}[h!]
    \centering
    \includegraphics[width=0.7\textwidth]{images/task_42/adc.png}
    \caption{Nearest-neighbors' average degree.}
    \label{fig:nn_ad_dist}
\end{figure}

\newpage